%!TEX encoding = UTF-8

\documentclass{article}
\usepackage[swedish]{babel}
\usepackage[utf8]{inputenc}

\begin{document}

\title{Daisy-D-Lite}
\author{Samuel Ytterbrink}
\date{\today}
\maketitle

\begin{abstract}
Daisy D Lite är en talboks spelare i HTML format.
den är utvecklad då HTML 5 standarden änu inte var färdig och har derför mer än bara programmmets dunktioner som svårigheter.
tanken bakom programet avr att var du än fanns skall du kunna lysna på DAISY böcker bara du har tillgång till din bok och en web läsare, och {\em Dasiy D Lite} förstås. 
Projektet har försök utnytja likheter i de båda standarderna och använad så lite kod för att gra så mycet som möjligt.
\end{abstract}

\section{inledning}

\section{Metod}

\subsection{Standader och protokoll}

\subsubsection{DAISY 2.02}
DAISY 2.02 togs fram redan 2001 och är det mest använda talboks formatet hos TPB\footnote{talboks och punktskrift biblioteket}. 
Denna standard bygger i sin tur på flera andra standarder, så som mp3, XHTML och SMIL.
Tillsammans tillför de möjligheten att orientera sig hirarkiskt och att syncronisera inehållet och där med bildar de en talbok.

\paragraph{XHTML}
De filer som är i XHTML formatet är bland annat den obligatoriska {\em ncc} filen.
I den filen finns meta information som bland annat vem som skrivit boken och vem som läser upp den.
I sidans kropp finns innehållsförteckningen för boken med rubriker och under rubriker, som {\em headings} 1 till 6.
I böcker där inte all text från orginal boken används, används oftast dessa rubriker som text till ljudet.
{\em span} tagen används för att markera att en ny sida har på börjats.

\paragraph{SMIL}
För att syncronisera data så används smil filer.
Dessa filler inehåller nestlade par\footnote{paralel} och seq\footnote{sequense} tagar där löven i trädet är text och audio taggar.
Det kan också finnas ett {\em Master} smil document som inehåller{\em ref} tagar som listar alla andra smil filer i spel ordning.

\paragraph{MP3}
även om man kan använda rå ljud data som PCM i form av wav filer så används nästan uteslutande mp3 filer.
När ljud kodas till mp3 

\subsubsection{HTML 5}
HTML 5 har vid skrivandets stund inte fast laggts änn men det finns en draft som under projecttets gång har förändrats.
olyckligt vis har dessutom delar som varit viktiga för projectet förändrats och lätt till en försvåring av utvecklingen.
Men utan dena draft hadde ju å andra sidan projectet inte ens kunnat påbörjas.
nedan 

\section{Diskussion}

\subsection{}

\section{Slutledning}

\end{document}
