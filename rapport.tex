%!TEX encoding = UTF-8

\documentclass{report}
\usepackage[swedish]{babel}
\usepackage[utf8]{inputenc}

\begin{document}

\title{Dasy-D-Lite}
\author{Samuel Ytterbrink}
\date{\today}
\maketitle

\begin{abstract}
Idag finns det ett problem för folk som har läs ock skriv svårigheter eller synfel som gör att de  behöver en dator till sin hjälp. 
De har inte den frihet att välja mellan de alternativ vanliga mäniskor har så som operativsystem eller plats att arbeta på. En anledning till deta är dyr programvara som inte går att använda på alla platformar och att det har blivit en standard att bara använda windows miljö i hjälpmedelsindustrin is sverige. På senare år har dock fri programvara och medvetna operativsystems tillverkare byggt in mer och mer i operativsystemen. Både Ubuntu, Mac OS och Windows 7 har förstoring som kan hjälpa de som har lätta problem. En sak som fortfarande fatas är stöd för speciel media så som indexerade talböker, så kallade DAISY böcker. Detta är vad detta project strävar efter att lössa.
\end{abstract}

\end{document}
