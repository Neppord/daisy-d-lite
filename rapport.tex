%!TEX encoding = UTF-8

\documentclass{report}
\usepackage[swedish]{babel}
\usepackage[utf8]{inputenc}

\begin{document}

\title{Daisy-D-Lite}
\author{Samuel Ytterbrink}
\date{\today}
\maketitle

\begin{abstract}

\end{abstract}

\section{inledning}
Idag finns det ett problem för folk som har läs ock skriv svårigheter eller synfel som gör att de  behöver en dator till sin hjälp. 
De har inte den frihet att välja mellan de alternativ vanliga människor har så som operativsystem eller plats att arbeta på.
En anledning till detta är dyr programvara som inte går att använda på alla plattformar och att det har blivit en standard att bara använda Windows miljö i hjälpmedelindustrin i Sverige.
På senare år har dock fri programvara och medvetna operativsystems tillverkare byggt in mer och mer i operativsystemen.
Både Ubuntu, Mac OS och Windows 7 har förstoring som kan hjälpa de som har lätta problem.
En sak som fortfarande fattas är stöd för speciell media så som indexerade talböcker, så kallade DAISY böcker.
Detta är vad detta projekt strävar efter att lösa.

Att genom små ock enkla medel se till att dessa böcker blir tillgängliga.

\section{Metod}

\subsection{Standader och protokoll}

\subsubsection{DAISY 2.02}

\subsubsection{HTML 5}

\section{Diskussion}

\subsection{}

\section{Slutledning}

\end{document}
